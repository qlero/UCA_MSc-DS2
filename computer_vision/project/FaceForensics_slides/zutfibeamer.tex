%%%%%%%%%%%%%%%%%%%%%%%%%%%%%%%%%%%%%%%%%%%%%%%%%%%%%%%%%%%%%%%%%%%%
%% I, the copyright holder of this work, release this work into the
%% public domain. This applies worldwide. In some countries this may
%% not be legally possible; if so: I grant anyone the right to use
%% this work for any purpose, without any conditions, unless such
%% conditions are required by law.
%%%%%%%%%%%%%%%%%%%%%%%%%%%%%%%%%%%%%%%%%%%%%%%%%%%%%%%%%%%%%%%%%%%%

% This theme was based on fibeamer theme 
% If you found any bugs please contact @karlosos
% This repository is hosted on github https://github.com/karlosos/zut-fibeamer/

\documentclass{beamer}
\usetheme[faculty=wi]{fibeamer}
\usepackage[utf8]{inputenc}
\usepackage[
  main=english,
  english
]{babel}

\includegraphics[width=0.5\textwidth]{fibeamer/logo/zut/logo.png}
\title{FaceForensics++, a seminal paper and dataset}
\subtitle{Quentin Le Roux, 2021}
% \author{Author's Name}

\usepackage{ragged2e}  % `\justifying` text
\usepackage{booktabs}  % Tables
\usepackage{tabularx}
\usepackage{tikz}      % Diagrams
\usetikzlibrary{calc, shapes, backgrounds}
\usepackage{amsmath, amssymb}
\usepackage{url}       % `\url`s
\usepackage{listings}  % Code listings
\usepackage{graphicx}

\frenchspacing
\begin{document}
  \frame[c]{\maketitle}
  \AtBeginSection[]{% Print an outline at the beginning of sections
    \begin{frame}<beamer>
      \frametitle{Outline for Section \thesection}
    %   \tableofcontents[currentsection]
    \end{frame}}

  \begin{darkframes}

    \begin{frame}[label=lists]{Context}
        % \framesubtitle{Subtitle}
        The Bushman Rock Shelter (BRS) site, Limpopo basin, South Africa, has yielded important 
        archaeological sequences from the Later Stone Age (LSA) prehistoric period (Porraz et al., 2018).\clearpage
        \centering\includegraphics[width=0.9\textwidth]{fibeamer/logo/zut/BRS.png}
    \end{frame}

    
    \begin{frame}[label=lists]{Conclusion}
        \textbf{Result}: wood charcoal SEM image classification with ML techniques is feasible even with a sparse dataset.\clearpage
        This is a proof of concept for the use of ML in archaeological science and archaebotany. It can be built upon by other, future teams.
        \clearpage
        To expand this work, further strategies should be explored such as:\begin{itemize}
        \item Balancing the data using a mix of under- and over-sampling.
        \item Using different model architectures
        \item Implement transfer learning
    \end{itemize}

    \end{frame} 

    \subsection{Citations and Bibliography}
    
    % \begin{frame}[label=bibliography]{Bibliography}
    %   \begin{thebibliography}{9}
    %     \scriptsize{
    %     \bibitem{ng}
    %         Ng, Andrew.
    %         \emph{NLP and Word Embeddings - CS230 Deep Learning, Stanford University}.
    %         deeplearning.ai, 2021.
    %     \bibitem{lecun}
    %         Le Cun, Yann.
    %         \emph{The MNIST database of handwritten digits}.
    %         1998.
    %     \bibitem{kipf2017semi}
    %         Kipf, Thomas N. and Welling, Max.
    %         \emph{Semi-Supervised Classification with Graph Convolutional Networks}.
    %         International Conference on Learning Representations (ICLR), 2017.
    %     \bibitem{battaglia2018relational}
    %         Battaglia, P. et al.
    %         \emph{Relational inductive biases, deep learning, and graph networks}. 2018.
    %     \bibitem{bruna2014spectral}
    %         Bruna, J. et al.
    %         \emph{Spectral Networks and Locally Connected Networks on Graphs}. 2014.
    %     }
    %   \end{thebibliography}
    % \end{frame}
    
\end{darkframes}

\end{document}